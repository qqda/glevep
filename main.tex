%!TEX root = ./main.tex
\documentclass[conference]{IEEEtran}
\IEEEoverridecommandlockouts
% The preceding line is only needed to identify funding in the first footnote. If that is unneeded, please comment it out.
\usepackage{cite}
\usepackage{amsmath,amssymb,amsfonts}
\usepackage{algorithmic}
\usepackage{graphicx}
\usepackage{textcomp}
\usepackage{xcolor}
\usepackage[utf8]{inputenc}
\usepackage{array}
\usepackage{gensymb}
\usepackage{mathtools}
\usepackage{units}
\usepackage{esvect}
\usepackage{multirow}
\usepackage{tikz}
\usepackage{subfig}
\usepackage{url}
\usepackage{listings}
\usepackage[ngerman]{babel}
\usepackage{enumerate}% http://ctan.org/pkg/enumerate

\newcommand{\code}{\texttt}
\renewcommand{\lstlistingname}{Quellcode}

\colorlet{punct}{red!60!black}
\definecolor{background}{HTML}{FFFFFF}
\definecolor{delim}{RGB}{20,105,176}
\colorlet{numb}{magenta!60!black}

\lstdefinelanguage{json}{
basicstyle=\normalfont\ttfamily,
numberstyle=\scriptsize,
stepnumber=1,
numbersep=8pt,
showstringspaces=false,
breaklines=true,
frame=lines,
backgroundcolor=\color{background},
literate=
*{0}{{{\color{numb}0}}}{1}
{1}{{{\color{numb}1}}}{1}
{2}{{{\color{numb}2}}}{1}
{3}{{{\color{numb}3}}}{1}
{4}{{{\color{numb}4}}}{1}
{5}{{{\color{numb}5}}}{1}
{6}{{{\color{numb}6}}}{1}
{7}{{{\color{numb}7}}}{1}
{8}{{{\color{numb}8}}}{1}
{9}{{{\color{numb}9}}}{1}
{:}{{{\color{punct}{:}}}}{1}
{,}{{{\color{punct}{,}}}}{1}
{\{}{{{\color{delim}{\{}}}}{1}
{\}}{{{\color{delim}{\}}}}}{1}
{[}{{{\color{delim}{[}}}}{1}
{]}{{{\color{delim}{]}}}}{1},
}


\lstdefinelanguage{epl}{
basicstyle=\normalfont\ttfamily,
numberstyle=\scriptsize,
stepnumber=1,
numbersep=8pt,
showstringspaces=false,
breaklines=true,
frame=lines,
backgroundcolor=\color{background}
}

\DeclarePairedDelimiter\ceil{\lceil}{\rceil}
\DeclarePairedDelimiter\floor{\lfloor}{\rfloor}

\def\BibTeX{{\rm B\kern-.05em{\sc i\kern-.025em b}\kern-.08em
    T\kern-.1667em\lower.7ex\hbox{E}\kern-.125emX}}
\begin{document}

%\title{TODO: globaleventprognosis}
%\title{Event Detection für Verhaltsmuster von Wikipedia-Nutzern}
%\title{Event Detection in Verhaltensmustern von Wikipedia-Nutzern}
%\title{Event Detection in Verhaltensmustern von Wikipedia-Artikeln}
\title{Event Detection in Wikipedia}


\author{\IEEEauthorblockN{Phillip Ginter}
\IEEEauthorblockA{\textit{Informatik (IN)} \\
\textit{Hochschule Furtwangen}\\
78120 Furtwangen, Deutschland \\
phillip.ginter@hs-furtwangen.de}
\and
\IEEEauthorblockN{Daniel Schönle}
\IEEEauthorblockA{\textit{Informatik (IN)} \\
\textit{Hochschule Furtwangen}\\
78120 Furtwangen, Deutschland \\
daniel.schoenle@hs-furtwangen.de}
}

\maketitle

% INPUT Content sections
%!TEX root = ./../main.tex
\begin{abstract}
Die freie Online-Enzyklopädie Wikipedia umfasst mehr als 74 Millionen Artikel, die anhaltend Änderungen unterzogen sind.
Die Motivation eine Nutzer für Änderungen kann unterschiedlich sein. Ereignisse der realen Welt, wie die Wahl des deutschen Bundeskanzlers, führen zu einem vorübergehenden Anstieg der Änderungen der entsprechenden Wikipedia-Artikel. 

Ziel dieser Arbeit ist es Ereignisse der realen Welt, anhand von Änderungen in Wikipedia-Artikeln sichtbar zu machen.
Zur Überwachung und Analyse der Artikeländerungen, wird eine Streaming-Data-Architektur entworfen und prototypisch umgesetzt.
Eine Analyse anhand modellierter Ereignis-Muster liefert dabei unzureichende Ergebnisse. Bessere Resultate wurden nach Erweiterung um Burst-Detection-Verfahren erzielt.



% Für die Erkennung wird ein Burst-Detection-Algorithmus genutzt, da die Mod v Mustern nicht die gewünschten Ergebnisse liefern konnte.

%     Das Ziel dieser Arbeit
% ist die Erkennung von Ereignissen der realen Welt, anhand von Änderungen an Wikipedia-Artikeln.

% Für die Überwachung und Analyse der Änderung an Wikipedia-Artikeln wird eine Streaming-Data-Architektur entworfen und prototypisch umgesetzt.

%     Die Modellierung von Mustern liefert dabei nicht die gewünschten Ergebnisse, weshalb der Einsatz eines Burst-Detection-Algorithmus vorgezogen wird.

%%TODO: Ziel erreicht, fehlt noch?
\end{abstract}

\begin{IEEEkeywords}
Wikipedia, Streaming-Data, Burst-Detection, Website-Activity-Tracking, Anomalieerkennung
\end{IEEEkeywords}
%!TEX root = ./../main.tex
\section{Einleitung}
Wikipedia ist eine freie Online-Enzyklopädie, mit dem Ziel „eine frei lizenzierte und hochwertige Enzyklopädie zu schaffen und damit lexikalisches Wissen zu verbreiten“ \cite{wales.}. Sie umfasst 74 Millionen Artikel, die von eine Commuity von 137.571 Nutzer erstellt und geprüft wurden. Seit 2001 wurden die Artikel 877.073.914 mal editiert, dies entspricht etwa 4.000.000 Edits pro Monat. \cite{wikistat}.

Aufgrund des kollaborativen Erstellungsprozesses können die Nutzer autonom und dynamisch Artikel ändern und erstellen. \cite{wikipedia.}
Die Gründe wesewegen die Nuzter aktiv werden variieren, einer davon ist die Dokumentation globaler Vorfälle. Dies können unter Anderen politische Veränderungen, Naturkatastrophen, sportliche Ereignisse oder Entwicklungen in der Vita prominenter Personen sein. Eine wichtige gemeinsame Eigenschaft ist eine hohe Relevanz für einen großen Nutzerkreis, der sich in vielen Aktivitäten, sogenannten Edits, niederschlägt. Interessant ist dabei auch welche Untermenge sich aus den aktiven Nutzer dafür bildet. Offensichtliche gemeinsame Eigenschaften dieser Nutzermengen sind geographische Verortung, Sprache, Kompetenzen und Interessen. 

Eine Untersuchung der Aktivitäten der Nutzer über Zeiträume hinweg und mittels statistisch gestützter Anaylse kann weitere Zusammenhänge herstellen \cite{10.1007978-3-642-36973-5_22}. Dafür gibt es zwei Herangehensweisen; eine Analyse gespeicherter Aktivitäten, sowie eine Beobachtung in Echtzeit. Georgescu et. al. haben in 'Extracting Event-Related Information from Article Updates in Wikipedia' ersteres durchgeführt, worauf im Abschnitt 'Verwandte Arbeiten' eingegangen wird. 

Für eine Anaylse in Echtzeit bieten sich eine Event-Driven-Architecture an. 

\begin{itemize}
    \item TODO
    \begin{itemize}
        \item Für eine Entität (z.\,B. eine Person des öffentlichen Lebens) aus der Gesamtheit der Wikipedia-Edit-Events in "Echtzeit" Events der realen Welt ableiten.
        \item Wir betrachten nur die Metadaten (Zeitstempel, Autor, ...) und nicht den Inhalt der Änderung Änderung (z.\,B. textuelle Änderung).
        \item ...
    \end{itemize}
    \item Wie sieht so ein Burst of Wikipedia-Edits aus \ref{fig:donald_rumsfelds_resignation_burst}?
\end{itemize}


\begin{figure}[h]
    \includegraphics[width=.5\textwidth]{images/Extracting_EventRelated_Information_from_Article.jpg}
    \caption{Donald Rumsfeld’s Rücktritt führte zu einem Burst an Autoren, die einen Wikipedia-Edit vornahmen \cite{10.1007978-3-642-36973-5_22}.}
    \label{fig:donald_rumsfelds_resignation_burst}
\end{figure}
\section{Verwandte Arbeiten}

\begin{figure*}
    \includegraphics[width=\textwidth]{images/eventdetect.png}
    \caption{Niagarino Anfragen der fünf Erkennungs-Methoden}
    \label{fig:eventdetect}
\end{figure*}

\subsection{Event-Driven Detection}
Weiler et. al. \cite{weiler2016evaluation} evaluieren fünf State of the Art Techniken zur Erkennung von noch unbekannten Ereignissen in einem Event Stream. Als Implementation wird Niagarino verwendet, eine von Weiler et. al. entwickelte Plattform die Apache Storm ähnelt. Die Techniken wurden als Anfragen umgesetzt, wie in Abbildung \ref{fig:eventdetect} dargestellt. Twitter-Events dienen als Datenbasis, die in einem Pre-Processing gefiltert werden. Dabei werden Retweets entfernt und der Inhalt der restlichen Tweets gesäubert, es verbleiben bibliographisch erkannte englische Wörter. 

\begin{itemize}
\item TopN\\Jedem Wort wird ein Wert zugewiesen, der auf dem Inverse-Document-Frequency des Zeitfenster beruht. Häufig vorkommende Wörter die im Allgemeinen selten verwendet werden erhalten die besere Bewertung. Nur die N-wichtigtsten Wörter verbleiben als Event.
\item Latent Dirichlet Allocation (LDA)\\Ist ein hierarchisches Bayes-Modell, das die Variation des Vokabulars in einer Gruppe von Dokumenten bewertet. Für jedes Zeitfenster extrahiert LDA  vermutete Ereignisse, die durch Ausdrücke beschrieben werden. 
\item Shifty\\Dabei werden die Veränderungen von TopN bewertet die sich zwischen zwei Zeitfenstern ereignen.
\item Event Detection with Clustering of
Wavelet-based Signals (EDCoW)\\Der Stream wird in Batches aufgeteilt, die dann per Wavelet-Analyse zu einem weiteren Event zusammengefasst werden. Dieses beinhaltet die Terme die im Zusammenhang mit dem erkannten Burst stehen, wovon unwichtige Wörter entfernt wurden. Durch Clustering enthält das Ergebnis Burst-Events die mit zwei Termen beschreiben werden.
\item Wavelet Analysis Topic Inference Summarization
(WATIS)\\ Eine Weiterentwicklung des (EDCoW), wobei nun zu jedem Burst fünf Terme geliefert werden.
\end{itemize}

\subsection{Burst Detection}
Da sich die Erkennung globaler Events mittels Esper als zumindest unvollständig erwies, wurden Methoden untersucht die das Verfahren  Ergänzen oder Ersetzen können.\\

Globale Events haben unter anderem auch die Eigenschaft eine erhöhte Aktivität im Edit-Event-Stream zu verursachen. Erhöhte Aktivität (Bursts) ist ein Phänomen das bereits seit Jahrzehnten untersucht wird, daher existieren eine Reihe von Theorien und abgeleiteten Algorithmen. Die zeitnahe Erstellung von Ergebnissen ist wichtig, weswegen auf eine schnelle Verarbeitung hin optimiert wird. Bursts besitzen eine Reihe an Attributen, die abhänig vom Use Case defininert werden. Dazu zählt der Zeitraum der jeweils auf das Auftreten eines Bursts untersucht wird, die Intensistät und die Verteilung der Intensistät auf den betrachteten Zeitraum.\\

Generell kann ein Datenstrom auf zwei Methoden betrachtet werden. Beim Point Monitoring wird nur das aktuelle Ereignis betrachtet. Liegt der Wert eines Attributs des Events in einen vordefinierten Bereich oder überschreitet es einen Grenzwert, wird ein Burst erkannt. Beim Aggregate Monitoring werden über einen Zeitraum (Windows) hinweg Ereignisse aggregiert. Drei Zeitfenstertypen werden unterschieden. Beim Landmark Window wird der Zeitraum durch einmalig fest gelegte Zeitpunkte definiert.
Das Sliding Window hingegen bewegt sich durch die Zeit, mit festgelegter Größe und festgelegtem Intervall. Die Parameter können als Zeitangaben oder als Zählangaben erfolgen. Ein Window wird entweder nach einer gewissen Zeit neu erstellt oder nach einer bestimmten Anzahl an Ereignissen. Beim Damped Window werden zudem jedem Ereignis Gewichtungen erteilt, je länger ein Ereignis zurückliegt, desto weniger Gewicht bekommt es mit Fortschreiten der Zeit. \cite{Zhu:2003:EEB:956750.956789}\\

\begin{figure*}[htbp]
\centerline{\includegraphics[height=4.4cm]{images/ratiopyramid.png}}
\caption{Von der klassischen 'Aggregataion Pyramid' zur 'Ratio Aggreation Pyramid' \cite{yuan2007online}}
\label{fig:ratiopyramid}
\end{figure*}


\begin{figure}[h]
    \includegraphics[width=.5\textwidth]{images/wavelet.jpg}
    \caption{Wavelet Tree \cite{Zhu:2003:EEB:956750.956789}}
    \label{fig:wavelet}
\end{figure}

Eine spezielles Modell entwickeln Zhu et. al. in 'Efficient Elastic Burst Detection in Data Streams' \cite{Zhu:2003:EEB:956750.956789} mit dem Elastic Window, bei dem   die Größe der Zeitfenster variiert. Somit können die Bursts dynamischer durch deren Charakteristik und unabhängiger von ihrer Dauer erkannt werden, die oftmals nicht im Vorfeld festgelegt werden kann. Dafür wird ein Wavelet-Tree vom Stream erzeugt, bei dem die Wavelet-Kooefizienten den Aggregaten der Fenster entsprechen. Wie in Abbildung \ref{fig:wavelet} zu sehen ist, wird zu Beginn der Stream in Fenster unterteilt, die disjunkt nebeneinander liegen. Je Fenster wird ein Aggregat erstellt - in der Regel werden dazu die Ereignisse eines bestimmten Typs oder Ereignisse mit gleichen Attributen gezählt. Diese Aggregate bilden nun selbst einen Stream. Das Verfahren wird wiederholt, vorausgesetzt es befinden sich entsprechende Ereignisse bzw. aggregierte Ereignisse im Stream. Mit jedem Schritt wird der betrachtete Zeitabschnitt größer und man bewegt in das nächst höhere Level Richtung Wurzel des Baums. Zu jedem Level wird ein Schwellwert angegeben, bei dessen Überscheitung ein Burst ausgelöst wird.\\

\begin{figure*}[htbp]
\centerline{\includegraphics[height=6.3cm]{images/slopepyramid.jpg}}
\caption{Berechnungsbeispiel zur 'Slope Pyramid' \cite{yuan2007online}}
\label{fig:slopepyramid}
\end{figure*}

% \begin{figure*}
%     \includegraphics[width.5=\textwidth]{images/ratiopyramid.jpg}
%     \caption{Von der klassischen 'Aggregataion Pyramid' zur 'Ratio Aggreation Pyramid' \cite{yuan2007online}}
%     \label{fig:ratiopyramid}
% \end{figure*}



Yuan et. al. entwickeln in 'Online Burst Detection Over High Speed Short Text Streams' \cite{yuan2007online} ein Verfahren das auf einem Wavelet-Tree basiert. Dieser kann auch als Pyarmide betrachtet werden, mit den Levels die dem Aggregierungsgrad entsprechen. Da die Burst-Detection sich auf den Schwellwert des jeweiligen Lebels stützt, muss dieser zu beginnder Analyse feststehen. Das kann eine Herausforderung sein, denn  die Menge an Ereignissen im Stream muss vor Beginn der Analyse eingeschätzt werden. Um dieses Problem zu lösen werden keine absoluten Angaben der Aggregatwerte verwendet, wie zum Beispiel deren Anzahl. Stattdessen wird angegeben, wieviele relevante Events aggregiert wurden im Verhältnis zu allen aufgetretenen Events im betroffenen Zeitraum. Abbildung \ref{fig:ratiopyramid}. Eine weitere Komprimierung wird durch die Slope Pyramid, wie in Abbildung \ref{fig:slopepyramid} dargestellt, erreicht. Dabei wird der Fokus auf die Veränderung von Fenster zu Fenster gelegt. Dazu wird der Verhältniswert aus der Ratio Pyamid herangezogen; der Wert des aktuellen Fensters wird durch den des vorhergehenden Fenster dividiert.\cite{yuan2007online}\\

Besondere Anforderungen stellt die Realtime-Burst Detection in Verbindung mit der gleichzeitigen Betrachtung mehrerer Fenstergrößen. Übliche Burst-Detektionsverfahren sind für eine Echtzeiterkennung nicht schnell genug, daher wurde von Ebina et. al. \cite{ebina2011real} Burst-Erkennungsmethode weiter entwickelt. Das Ziel ist dabei die Berechnungszeit zu reduzieren, indem redundante Datenaktualisierungen wie bei Yuan et. al. vermieden werden. Dazu werden die Zellen der Slope Pyramid auf effizientere Weise berechnet.\cite{ebina2011real}\\

Je nach Use Case sind auch aufwändigere Analysen wie in 'Event Detection with Burst Information Networks' \cite{ge2016event} notwendig.

% Detecting ‘bursts’ in time series data with Kleinberg’s burst detection algorithm
% \cite{kleinberg1}

\subsection{Social Media Analyse}
Eine ganze Reihe von Arbeiten beschäftigt sich mit der Analyse der Aktivitäten der Wikipedia-Nutzer.

\begin{itemize}
\item Extracting Event-Related Information from Article Updates in Wikipedia \cite{10.1007978-3-642-36973-5_22}
\item Ongoing events in Wikipedia: a cross-lingual case study \cite{gottschalk2017ongoing}
\item How much is Wikipedia Lagging Behind News? \cite{fetahu2015much}
\item Event analysis in social multimedia: a survey \cite{liu2016event}
\item A cloud-enabled automatic disaster analysis system of multi-sourced data streams: An example synthesizing social media, remote sensing and Wikipedia data \cite{huang2017cloud}
\end{itemize}

% Des weiteren 

% Pinterest
% I need to try this?: a statistical overview of pinterest
% \cite{gilbert2013need}

% Twitter
% An evaluation of the run-time and task-based performance of event detection techniques for Twitter
% \cite{weiler2016evaluation}

%!TEX root = ./../main.tex
\section{Streaming Data}

Das Ziel unserer Aufgabenstellung ist die Verarbeitung von Streaming-Daten in Echtzeit. Die Streaming Data-Architektur
von Psaltis \cite{psaltis2017streaming} ist für diese Art von Problem konzipiert und bildet die Grundlage für
unsere Architektur. Nach der kurzen Vorstellung der Streaming Data-Architektur von Psaltis, zeigen wir unsere eigene konkrete
Umsetzung und vergleichen eingesetzten Technologien mit Alternativen.

\subsection{Streaming Data-Architektur nach Psaltis}

In Abbildung \ref{fig:streaming_data_architecture-psaltis} sind die Komponenten der Streaming Data-Architektur, wie Psaltis \cite{psaltis2017streaming}
sie entwickelt hat, abgebildet. Der Collection Tier ist der Einstiegspunkt, der Daten in das System bringt. Unabhängig von dem verwendeten
Protokoll, werden die Daten mithilfe eines dieser Patterns übertragen \cite{psaltis2017streaming}:
\begin{itemize}
    \item Request/response pattern
    \item Publish/subscribe pattern
    \item One-way pattern
    \item Request/acknowledge pattern
    \item Stream pattern
\end{itemize}

Bei der Datenquelle kann es sich sowohl um von Hardware und auch von Software generierte Events handeln. Beispiele hierfür sind
Temperatur, Lautstärke oder auch Browser-Clicks.

\begin{figure}[h]
    \includegraphics[width=.5\textwidth]{images/streaming_data_architecture-psaltis.jpg}
    \caption{Streaming Data-Architektur \cite{psaltis2017streaming}}
    \label{fig:streaming_data_architecture-psaltis}
\end{figure}

Um die Daten vom Collection Tier auf den Rest der Pipeline zu verteilen wird ein Messaging Queueing Tier implementiert.
Das Message Queueing Model ist ein Modell zur Interprozesskommunikation. Anwendungen schicken
Nachrichten an eine Nachrichtenschlange, von der andere Anwendungen diese abholen können \cite{gray2003interprocess}.
Dadurch werden die eingesetzten Systeme voneinander entkoppelt und die Kommunikation findet nicht durch direkte Aufrufe,
sondern über die Queue statt \cite{psaltis2017streaming}. Im vorliegenden Fall wird der Collection Tier vom Rest der Pipeline entkoppelt.

Im Analysis Tier werden auf die Streams des Message Queueing Tiers kontinuierlich Queries angewandt um Muster in den Daten zu erkennen.
Hier können Event Stream Processing-Systeme (ESP) wie z.\,B. Apache Flink, Apache Spark oder Apache Storm oder
auch Complex Event Processing-Systeme (CEP) wie z.\,B. Esper oder Apache Samza eingesetzt werden. \cite{psaltis2017streaming}
In ESP wird die Verarbeitungslogik imperativ umgesetzt, wohingegen in CEP einen deklarativen Ansatz verfolgt.
Die Event Processing Languages (EPL) sind Sprachen, die über Sequenz-, Konjunktions-, Disjunktions- und Negationsoperatoren
verfügen und für CEP entwickelt wurden \cite{hedtstck2017complex}. Als Basis der Operatoren werden Windows eingesetzt.
Ist die Analyse vorbei und die Ergebnisse stehen fest, können diese verworfen werden, zurück in die Streaming-Plattform gespeichert werden,
für eine Echtzeitnutzung oder die Stapelverarbeitung gespeichert werden \cite{psaltis2017streaming}. Wie der Abbildung \ref{fig:streaming_data_architecture-psaltis}
zu entnehmen ist, schlägt Psaltis hierfür eine In-Memory-Datenbank (IMDB), einen Data Access Tier und ein Long Term Storage vor.

\subsection{Unsere Streaming Data-Architektur}
In diesem Kapitel geht es um die Frage, welche Technologie wir in dem jeweiligen Tier einsetzen und wieso wir uns dafür entschieden haben.
Technische Details, die die Implementierung betreffen, erläutern wir im nächsten Kapitel.
TODO: Quellen zu den einzelnen Punkten/Systemen
\begin{itemize}
    \item \textit{Collection Tier.} Als Datenquelle haben wir Daten von Wikipedia. Konkret handelt es sich um die RecentChanges,
    also die aktuellen Änderungen an Wikipedia-Einträgen. Details hierzu sind im nächsten Kapitel beschrieben. Wir entschieden uns für Wikipedia als
    Datenquelle, weil es eine offene Plattform ist, die einen freien Zugang zu allen Daten gibt.

    \item \textit{Messaging Queuing Tier.} Wir nutzen Kafka als Messaging-System, weil es skalierbar ist, einen hohen Datendurchsatz ermöglicht,
    Real-Time-Messaging unterstützt, eine einfache Client-Anbindung ermöglicht und Events standardmäßig persistiert. Der letzte Punkt ist
    besonders relevant, da für die Entwicklung der Regeln eine statische Datenmenge einfacher zu handhaben ist.

    \item \textit{Analysis Tier.} Die Analyse der Wikipedia-EditEvents sollte deklarativ erfolgen. Dadurch erhofften wir uns
    in kurzer Zeit eine Vielzahl an Regeln zu testen und so ein gutes Verständnis für die Daten zu erhalten. Aufgrund dieser Vorgabe
    entschieden wir uns für Esper. Es bietet eine CEP-Implementierung, die nach einem regelbasierten Ansatz (somit deklarativ) arbeitet.
    Ein weiterer Punkt, der für Esper sprach, war zum einen die Esper Processing Language und die Implementierung der Anwendung in Java.

    \item \textit{In-Memory Data Store, Long Term Storage und Data Access Tier.} Für diese drei Komponenten haben wir selbst keine Implementierung vorgesehen.
    Im Ausblick geben wir hierzu Ideen zu möglichen Erweiterungen.
\end{itemize}
%!TEX root = ./../main.tex
\section{Prototyp}
Zur Lösung der Aufgabenstellung haben wir einen lauffähigen Prototypen entwickelt, der die Machbarkeit demonstriert.
Dafür haben wir die im vorhergehenden Kapitel genannten Technologien eingesetzt. Die Details zu den jeweils entstandenen
Anwendungen stellen wir in diesem Kapitel vor.

\begin{figure*}
    \includegraphics[width=\textwidth]{images/Architektur_Uebersicht.png}
    \caption{Architekturübersicht}
    \label{fig:architektur_uebersicht}
\end{figure*}

Abbildung \ref{fig:architektur_uebersicht} zeigt die Teilsysteme und deren Interaktion untereinander.

% Abbildung \ref{fig:architektur_uebersicht} zeigt bisher nur eine sehr rudimentäre Übersicht und soll als Grundlage dienen.
% Die Verbindung zwischen Esper und Kafka muss noch genauer werden: Protokoll, senden von komplexen Events in neue Kafka-Topics.

\subsection{Collection Tier: Wikipedia}
Wie zuvor beschrieben, stützt sich unser System auf Daten von Wikipedia. Wir nutzen den RecentChanges-Stream, der
Events zu neu erstellten, aktualisierten und gelöschten Wikipedia-Seiten enthält\footnote{https://www.mediawiki.org/wiki/Manual:RCFeed}.
Der Quellcode \ref{WikipediaRecentChangeJSON} zeigt einen Ausschnitt eines solchen Events.
Es sind nur die Attribut abgebildet, die in einem der nachfolgenden Teilsysteme relevant sind.

\begin{lstlisting}[label=WikipediaRecentChangeJSON,caption=Wikipedia RecentChange-Event,language=json,firstnumber=1,captionpos=b]
{
    "bot": false,
    "comment": "/* Politischer Werdegang */",
    "id": 269108829,
    "meta": {
        "uri": "https://de.wikipedia.org/wiki/Ingeborg_H%C3%A4ckel",
        "partition": 0,
        "offset": 1329388977
    },
    "timestamp": 1547910734,
    "title": "Ingeborg H\"ackel",
    "user": "Eszet2000",
    "wiki": "dewiki"
}
\end{lstlisting}

Die beiden Parameter \code{offset} und \code{uri}, innerhalb des \code{meta}-Objektes, stammen aus einem Kafka-System
und dienen der Wiederaufnahme eines abgebrochenen Streams. Wikipedia setzt intern Apache Kafka als Messaging-System ein.
Nach außen nutzt Wikipedia EventStreams. Das ist ein Webservice der kontinuierliche Datenströme mit strukturierten Daten
über HTTP sendet\footnote{https://wikitech.wikimedia.org/wiki/EventStreams}. Die Basis-Technologie dessen, ist
das Server-Sent Event (SSE) Protokoll. Der RecentChanges-Stream ist ein solcher Stream und kann über eine Client-Bibliothek
konsumiert werden.

Wie im vorherigen Kapitel beschrieben, werden Daten vom Collection Tier mithilfe eines der aufgelisteten Pattern übertragen.
SSE funktioniert nach dem One-way pattern, indem eine HTTP-Verbindung aufgemacht wird um Nachrichten zu empfangen \cite{EventSource_SSE}.

\subsection{Implementierungsdetails zum Messaging Queuing Tier: Kafka}
Im Messaging Queuing Tier setzen wir Apache Kafka in der Version 2.1 ein, um die im Collection Tier beschriebenen Wikipedia-Events
in unser eigenes Messaging-System zu überführen. Hierfür haben wir eine Java-Anwendung entwickelt, in der die folgenden
Schritte nacheinander ausgeführt werden:
\begin{enumerate}
    \item \textit{Kafka Initialisierung.} Den Host des Bootstrap-Servers setzen um eine Verbindung zu erzeugen.
    Als Key- und Value-Serialisierer setzen wir
    jeweils den \code{StringSerializer} von Kafka ein. Das heißt, die Events werden als JSON-String in das Topic \code{wikiEdit}
    eingespeist. Zum Senden von Events erzeugen wir ein Producer-Objekt mit dem passenden Typ \code{Producer<String, String>}.
    An dieser Stelle war die Überlegung, anstelle eines String-Serialisierers für den Wert des Producers,
    einen eigenen Serialisierer für die WikipediaEditEvent-Klasse einzusetzen. Wir entschieden uns aber für den
    String-Serialisierer, da Wikipedia die Events auch als JSON-String überträgt und wir in Kafka selbst
    keine weiteren Operationen an den Daten vornehmen. Eine mögliche Operation könnte ein Filter sein.
    Aber das einzige Ziel von Kafka soll die Persistierung und Weitergabe von Daten sein und dafür ist keine Serialisierung in
    einem bestimmten Typen notwendig. Außerdem verschiebt es Komplexität aus der Kafka-Anwendung
    in die Consumer-Anwendungen.
    \item \textit{Erzeugung eines EventHandlers.} Für das Empfangen von EventSource-Nachrichten nutzen wir die Java-Bibliothek
    \textit{okhttp-eventsource}\footnote{https://github.com/launchdarkly/okhttp-eventsource}. Zur Verarbeitung der Events
    \code{onOpen}, \code{onClose}, \code{onMessage}, \code{onComment} und \code{onError} muss das Interface \code{EventHandler} von
    \textit{okhttp-eventsource} implementiert werden.
    \item \textit{Erzeugung und Starten einer EventSource.} Mit der Stream-URI der Wikipedia-EventSource (für die RecentChanges ist das: https://stream.wikimedia.org/v2/stream/recentchange)
    und des implementierten EventHandler-Interfaces
    kann ein \code{EventSource}-Objekt erzeugt werden. Das Objekt dient dem Starten und Beenden eines EventSource-Streams.
    Die Daten werden dann, wie zuvor beschrieben, durch das SSE-Protokoll von Wikipedia an die Anwendung gesendet.
    \item \textit{Beim Eintreffen eines Events: Senden einer Nachricht in ein Kafka-Topic.} Tritt ein Wikipedia-Event auf,
    wird die \code{onMessage}-Methode des implementierten EventHandlers-Interface aufgerufen. Der zweite Parameter enthält die Daten
    des aufgetretenen Events. Der Zugriff auf die als JSON-String codierte Nachricht erfolgt über die \code{getDate()}-Methode.
    Diese Daten sendet die Anwendung, über den zuvor erzeugten Producer, ohne eine weitere
    Verarbeitung in das Kafka-Topic \code{wikiEdit}.
\end{enumerate}

Die Konfiguration der Kafka-Topics ist sehr einfach gehalten, da es sich bei der Anwendung nur um einen Prototypen handelt.
Wir haben ein Topic mit dem Namen \code{wikiEdit}. Da die Anwendung auf nur einem Server läuft, setzen wir auch nur
eine Partition ein und haben keine Replikation. Eine Skalierung der Kafka-Anwendung auf mehrere
parallel-arbeitende Server ist jedoch nicht ausgeschlossen für eine Produktivanwendung.

\subsection{Implementierungsdetails zum Analysis Tier: Esper}
In unserer Esper-Anwendung, die das Hauptsystem des Analysis Tier ist, nutzen wir Esper in Version 7.1
als Complex Event Processing-Werkzeug.
Zur Verarbeitung der Wikipedia-Events haben wir zwei Lösungen umgesetzt, da die erste Lösung nicht
zum Erreichen der Ziele führte. Für ein besseres Verständnis geben wir einen kurzen Überblick über den gemeinsamen Aufbau beider
Lösungen. Danach beschreiben wir die Details und Unterschiede der jeweiligen Lösungen und analysieren diese hinsichtlich der
Zielerfüllung.
In unserer Esper-Anwendung sind wir wie folgt vorgegangen:

\begin{enumerate}
    \item \textit{Esper Initialisierung.} Die Initialisierung von Esper besteht aus der Erzeugung einer \code{Configuration},
    der Erstellung und dem Starten von EPL-Statements, sowie dem Erzeugen von Listener-Klassen.
    \item \textit{Kafka initialisieren und starten.} Um die Daten aus dem Messaging Queuing Tier zu bekommen haben wir einen
    \code{KafkaConsumer<String, String>} implementiert.
    Wir pollen in einer Endlosschleife alle 10 Millisekunden die Daten vom Kafka-Topic \code{wikiEdit}. Bei den empfangenen Daten
    handelt es sich um einen String, der JSON enthält. Mithilfe der Gson-Bibliothek\footnote{https://github.com/google/gson}
    konvertieren wir die JSON-Strings in Java-Objekte. Die daraus resultierenden Java-Objekte senden wir wiederum in das
    Esper-System, damit darauf die EPL-Statements angewandt werden können.
    \item \textit{Aktion ausführen, sobald das Muster erfüllt ist.} Ist das Muster erfüllt, wird die \code{update}-Methode der Listener-Klasse
    aufgerufen. Es werden die alten und neuen Events übergeben. Daraus kann eine Aktion erfolgen, z.\,B. dass erzeugen eines neuen
    komplexen Events.
\end{enumerate}

Abbildung \ref{fig:class_diagram_eventtypes}


\begin{figure}[h]
    \includegraphics[width=.5\textwidth]{images/Esper1.png}
    \caption{Klassendiagramm der zwei Ereignistypen WikipediaEditEvent, WikipediaEditEventFive und der Helferklasse Meta}
    \label{fig:class_diagram_eventtypes}
\end{figure}


Genaueres zu den erzeugten Expressions im nächsten Kapitel.

Einsatz von GSON

\subsection{BurstDetection-Implementierung}
\section{Anwendungsfälle}
% https://wikitech.wikimedia.org/wiki/EventStreams/Powered_By
Schauen, was man aus dem Paper bekommt: \cite{10.1007978-3-642-36973-5_22}

\begin{enumerate}
    \item
    Voraussetzung: 2 oder mehr Autoren (die kein Bot sind) bearbeiten ein
    Ergebnis:
\end{enumerate}




\section{Praktische Analyse}

\begin{itemize}
    \item Wie sieht ein konkretes Wikipedia-Edit-Event aus / aus welchen Bestandteilen besteht es?
        siehe \cite{10.1007978-3-642-36973-5_22} Kapitel 2 Anfang
    \item Anhand von Burst Detection, wollen wir Events der realen Welt ableiten \cite{Zhu:2003:EEB:956750.956789}
    \item Mit welchen Expressions decken wir welche Use Cases ab?
    \item Wie sind wir auf die Expressions gekommen? Nur durch ausprobieren?
    \item Reale Beispiele für "passende" Events
    \item Welche neuen "Komplexen Events" erzeugen wir?
    \item Reale Beispiele für komplexe Events
    \item Welche Ergebnisse liefert das System?
    \item Übersicht der Hierarchie von Ereignistypen: siehe EP\_5\_CEP\_1\.pdf
\end{itemize}
%!TEX root = ./../main.tex
\section{Ergebnisse}
- Der Einsatz von Esper im Analysis Tier war für anfängliche Versuche richtig.
Mit zunehmender Komplexität bekamen wir hierdurch Schwierigkeiten. Durch die Abstraktion, die Esper bietet konnten
wir beispielsweise nicht auf vergangene Events zugreifen. Die Funktionalität, die in Flink unterstützt und dort ...

\section{Diskussion}
aa
\section{Ausblick}

\section{Beiträge der Autoren}
 Phillip Ginter und Daniel Schönle haben gleichermaßen zu dieser Arbeit beigetragen und sind Erstautoren.

% Quellenverzeichnis
\bibliographystyle{IEEEtran}
\bibliography{bibtex}

\end{document}
