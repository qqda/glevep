%!TEX root = ./../main.tex
\section{Streaming Data}

Das Ziel unserer Aufgabenstellung ist die Verarbeitung von Streaming-Daten. Die Streaming Data-Architektur
von Psaltis \cite{psaltis2017streaming} ist für diese Art von Problem konzipiert und bildet die Grundlage für
unsere Architektur. Nach der Vorstellung der Streaming Data-Architektur von Psaltis, zeigen wir unsere eigene konkrete
Umsetzung und vergleichen eingesetzten Technologien mit Alternativen.

\subsection{Streaming Data-Architektur}
\begin{itemize}
    \item Wie sieht eine Streaming Data-Architektur aus bzw. aus welchen Komponenten besteht sie \cite{psaltis2017streaming}?
    \item Ist eine Streaming Data-Architektur für unsere Aufgabenstellung sinnvoll?
\end{itemize}

\subsection{Konkrete Implementierung unserer Streaming Data-Architektur}
Die obigen vier Stufen der Streaming Data-Architektur von Psaltis \cite{psaltis2017streaming}
haben wir mit den folgenden konkreten Technologien gefüllt, um die Aufgabenstellung aus dem vorhergehenden Kapitel zu erfüllen.
In diesem Kapitel geht es um die Frage, wieso wir uns für konkrete Technologien entschieden haben und im nachfolgenden Kapitel
stellen wir die Implementierungsdetails vor.
\begin{itemize}
    \item \textit{Collection tier.} Wikipedia ist unsere Datenquelle.
    \item \textit{Messaging queuing tier.} Wir nutzen ein Kafka-System: Weshalb Kafka? Welche Features
        (Durable messaging, Different Messaging Systems, Scalability, Performance, Transaction Support, Security, ...) sind für uns von großer Relevanz?
        Oder soll Kafka ein eigenes Kapitel bekommen?
    \item \textit{Analysis tier.} Esper: warum / welche Features sind für uns von Relevanz? Wie sieht die Ausgabe nach der Analyse aus?
    \item \textit{Data access tier.} ???
\end{itemize}