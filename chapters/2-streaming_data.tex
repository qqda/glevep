%!TEX root = ./../main.tex
\section{Streaming Data}
Folgendes könnte man in diesem Kapitel behandeln:

\begin{itemize}
    \item Wie sieht eine Streaming Data-Architektur aus bzw. aus welchen Komponenten besteht sie \cite{psaltis2017streaming}?
    \item Ist eine Streaming Data-Architektur für unsere Aufgabenstellung sinnvoll?
    \item Falls ja. Wie ist unsere konkrete Streaming Data-Architektur aufgebaut und wieso?
\end{itemize}



\subsection{Streaming Data-Architektur}
Punkt 1 und 2 von der obigen Liste hier beschreiben mit \cite{psaltis2017streaming}.



\subsection{Konkrete Implementierung unserer Streaming Data-Architektur}
Die obigen vier Stufen einer Streaming Data-Architektur von Psaltis \cite{psaltis2017streaming}
haben wir mit den folgenden konkreten Inhalten gefüllt, um die Aufgabenstellung aus dem vorhergehenden Kapitel zu erfüllen.
In diesem Kapitel geht es um die Frage, wieso wir uns für konkrete Technologien entschieden haben und im nachfolgenden Kapitel
geht es um die Implementierungsdetails dieser Technologien.
\begin{itemize}
    \item \textit{Collection tier.} Wikipedia ist unsere Datenquelle.
    \item \textit{Messaging queuing tier.} Wir nutzen ein Kafka-System: Weshalb Kafka? Welche Features
        (Durable messaging, Different Messaging Systems, Scalability, Performance, Transaction Support, Security, ...) sind für uns von großer Relevanz?
        Oder soll Kafka ein eigenes Kapitel bekommen?
    \item \textit{Analysis tier.} Esper: warum / welche Features sind für uns von Relevanz? Wie sieht die Ausgabe nach der Analyse aus?
    \item \textit{Data access tier.} ???
\end{itemize}