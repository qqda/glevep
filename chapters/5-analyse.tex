\section{Anwendungsfälle}
\subsection{Erkennung globler Events}
Die Erkennung global relevanter Events wurde bereits in \ref{sec:Einleitung} beschreiben.

Der Fokus soll dabei auf der Verarbeitung der Events in Echtzeit, dem Erkennen von Ereignismustern im Wikipedia-Edit-Eventsstrom und dem generieren von komplexen Events liegen. Eine komplexe Datenanalyse nicht vorgesehen, statt dessen soll durch eine einfache Mustererkennung genutzt werden. Dabei werden Events innerhalb von Zeitfenstern aggregiert und in Beziehung zueinander gesetzt um Rückschlüsse auf 'Global Events' zu ziehen.

Die Zielsetzung, den Fokus auf das Erkennen von Ereignismustern in Wikipedia-Edit-Eventsstrom zu setzen und dabei auf komplexe Datenanalyse zu verzichten, schränkt die Erfolgsrate ein.

Ebenso kann in der Realtime-Analyse ein Hinzuziehen externer Informationen nicht erfolgen. Deswegen sind Abfragen externer Quellen, eine Textanalyse der Änderungen auf die in den Edits verweisen wird, eine Verortung der Seite in der Wikipedia-Hierarchie und das Parsen externer Web-Seiten zum Zweck der weiteren Informationsgewinngung nicht möglich. 

% Einfache Muster können in Echtzeit erkannt werden. Bei erfolgreicher Erkennung dieser Muster werden komplexe Events erzeugt. Dadurch erhöht sich die Informationsdichte im Stream und vergrößert die Granularität, verringt die Frequenz. Aufwändigeren Analysen werden daher an komplexe Events gebunden.


\subsection{Entworfene Anwendungsfälle}
Die Analyse des Wikipedia-Edit-Streams wurde auf weitere Anwendungsfälle ertweitert.

\begin{itemize}
    \item Edit-Wars Detection\\ Bei Edit-Wars stimmen die Vorstellungen über den Inhalt des Artikels der Autoren nicht überein. Die Folge sind wiederholtes Revidieren der Änderungen, häufig in hoher Frequenz.
    \item Fraud Detection\\Ab wann Vandalismus \\Vorrausgesetzt Muster, die sich temporal entwickeln
    \item Load Prediction\\ TODO
    \item Semantic-Clustering\\ TODO
\end{itemize}
