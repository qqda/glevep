\section{Anwendungsfälle}
\subsection{Entworfene Anwedungsfälle}
Die Zielsetzung, den Fokus auf das Erkennen von Ereignismustern in Wikipedai-Edit-Eventsstrom zu setzen, schränkt die Komplexität der Datenanlyse ein. Ebenso ist ein Hinzuziehen externer Informationen mit hohem Aufwand verbunden. Deswegen sind Abfragen externer Quellen, Textanalyse der Änderungen, Verortung der Seite in der Wikipedia-Hierarchie und Parsen externer Web-Seiten nur in geringer Frequenz möglich. Die Bildung komplexer Events erhöht der Informationsdichte und vergrößert die Granularität, verringt die Frequenz. Aufwändigeren Analysen werden daher an komplexe Events gebunden.

\begin{itemize}
    \item Global Event Detection\\ TODO
    \item Edit-Wars Detection\\ Bei Edit-Wars stimmen die Vorstellungen über den Inhalt des Artikels der Autoren nicht überein. Die Folge sind wiederholtes Revidieren der Änderungen, häufig in hoher Frequenz.
    \item Fraud Detection\\Ab wann Vandalismus \\Vorrausgesetzt Muster, die sich temporal entwickeln
    \item Load Prediction\\ TODO
    \item Semantic-Clustering\\ TODO
\end{itemize}


% https://wikitech.wikimedia.org/wiki/EventStreams/Powered_By
Schauen, was man aus dem Paper bekommt: \cite{10.1007978-3-642-36973-5_22}

\begin{enumerate}
    \item
    Voraussetzung: 2 oder mehr Autoren (die kein Bot sind) bearbeiten ein
    Ergebnis:
\end{enumerate}




\section{Praktische Analyse}

\begin{itemize}
    \item Wie sieht ein konkretes Wikipedia-Edit-Event aus / aus welchen Bestandteilen besteht es?
        siehe \cite{10.1007978-3-642-36973-5_22} Kapitel 2 Anfang
    \item Anhand von Burst Detection, wollen wir Events der realen Welt ableiten \cite{Zhu:2003:EEB:956750.956789}
    \item Mit welchen Expressions decken wir welche Use Cases ab?
    \item Wie sind wir auf die Expressions gekommen? Nur durch ausprobieren?
    \item Reale Beispiele für "passende" Events
    \item Welche neuen "Komplexen Events" erzeugen wir?
    \item Reale Beispiele für komplexe Events
    \item Welche Ergebnisse liefert das System?
    \item Übersicht der Hierarchie von Ereignistypen: siehe EP\_5\_CEP\_1\.pdf
\end{itemize}