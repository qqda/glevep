%!TEX root = ./../main.tex
\begin{abstract}
Die freie Online-Enzyklopädie Wikipedia umfasst mehr als 74 Millionen Artikel, die anhaltend Änderungen unterzogen sind.
Eine Motivation der Nutzer, Änderungen an Artikeln durchzuführen, ist die Dokumentation von weltbewegenden Veränderungen. Ereignisse, wie die Wahl des deutschen Bundeskanzlers, führen vorübergehend zu häufigen Änderungen an den entsprechenden Wikipedia-Artikeln. 

Es ist Ziel dieser Arbeit Ereignisse der realen Welt anhand von Änderungen in Wikipedia-Artikeln sichtbar zu machen.
Zur Überwachung und Analyse von Artikeländerungen, wird eine Streaming-Data-Architektur entworfen und prototypisch umgesetzt.
Eine Analyse anhand modellierter Ereignis-Muster liefert dabei unzureichende Ergebnisse. Bessere Resultate wurden unter Einbeziehung von Burst-Detection-Verfahren erzielt.



% Für die Erkennung wird ein Burst-Detection-Algorithmus genutzt, da die Mod v Mustern nicht die gewünschten Ergebnisse liefern konnte.

%     Das Ziel dieser Arbeit
% ist die Erkennung von Ereignissen der realen Welt, anhand von Änderungen an Wikipedia-Artikeln.

% Für die Überwachung und Analyse der Änderung an Wikipedia-Artikeln wird eine Streaming-Data-Architektur entworfen und prototypisch umgesetzt.

%     Die Modellierung von Mustern liefert dabei nicht die gewünschten Ergebnisse, weshalb der Einsatz eines Burst-Detection-Algorithmus vorgezogen wird.

%%TODO: Ziel erreicht, fehlt noch?
\end{abstract}

\begin{IEEEkeywords}
Wikipedia, Streaming-Data, Burst-Detection, Website-Activity-Tracking, Anomalieerkennung
\end{IEEEkeywords}