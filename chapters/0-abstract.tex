%!TEX root = ./../main.tex
\begin{abstract}
Die freie Online-Enzyklopädie Wikipedia umfasst mehr als 74 Millionen Artikel, die anhaltend Änderungen unterzogen sind.
Die Motivation eine Nutzer für Änderungen kann unterschiedlich sein. Ereignisse der realen Welt, wie die Wahl des deutschen Bundeskanzlers, führen zu einem vorübergehenden Anstieg der Änderungen der entsprechenden Wikipedia-Artikel. 

Ziel dieser Arbeit ist es Ereignisse der realen Welt, anhand von Änderungen in Wikipedia-Artikeln sichtbar zu machen.
Zur Überwachung und Analyse der Artikeländerungen, wird eine Streaming-Data-Architektur entworfen und prototypisch umgesetzt.
Eine Analyse anhand modellierter Ereignis-Muster liefert dabei unzureichende Ergebnisse. Bessere Resultate wurden nach Erweiterung um Burst-Detection-Verfahren erzielt.



% Für die Erkennung wird ein Burst-Detection-Algorithmus genutzt, da die Mod v Mustern nicht die gewünschten Ergebnisse liefern konnte.

%     Das Ziel dieser Arbeit
% ist die Erkennung von Ereignissen der realen Welt, anhand von Änderungen an Wikipedia-Artikeln.

% Für die Überwachung und Analyse der Änderung an Wikipedia-Artikeln wird eine Streaming-Data-Architektur entworfen und prototypisch umgesetzt.

%     Die Modellierung von Mustern liefert dabei nicht die gewünschten Ergebnisse, weshalb der Einsatz eines Burst-Detection-Algorithmus vorgezogen wird.

%%TODO: Ziel erreicht, fehlt noch?
\end{abstract}

\begin{IEEEkeywords}
Wikipedia, Streaming-Data, Burst-Detection, Website-Activity-Tracking, Anomalieerkennung
\end{IEEEkeywords}