%!TEX root = ./../main.tex
\begin{abstract}
Die freie Enzyklopädie Wikipedia umfasst über 74 Millionen Artikel, die dauerhaft Änderungen unterzogen sind.
Auslöser für diese Änderungen können vielfältig sein. Ereignisse der realen Welt wie z.\,B. die Wahl des deutschen Bundeskanzlers
    führen zu einem starken kurzzeitigen Anstieg der Änderungen an dem entsprechenden Wikipedia-Artikel. Das Ziel dieser Arbeit
ist die Erkennung von Ereignissen der realen Welt, anhand von Änderungen an Wikipedia-Artikeln.
Für die Überwachung und Analyse der Änderung an Wikipedia-Artikeln wird eine Streaming-Data-Architektur entworfen und prototypisch umgesetzt.
    Die Modellierung von Mustern liefert keine konkreten Ergebnisse, weshalb der Einsatz eines Burst-Detection-Algorithmus vorgezogen wird.
Ziel erreicht
\end{abstract}

\begin{IEEEkeywords}
wikipedia, prognosis, global event
\end{IEEEkeywords}