\section{Verwandte Arbeiten}
\subsection{Burst Detection}

Online Burst Detection Over High Speed Short Text Streams
\cite{yuan2007online}


Event Detection with Burst Information Networks
\cite{ge2016event}

Realtime
Die Realtime-Burst Detection über mehrere Fenstergrößen ist für die Analyse von Datenströmen hilfreich. Die üblichen Burst-Detektionsverfahren sind für die Echtzeiterkennung nicht effektiv. Die Realtime-Burst Detection benötigt eine neue Burst-Erkennungsmethode, die die Berechnung reduziert, indem redundante Datenaktualisierungen vermieden werden. Dabei wird ein Ereignis bei seinem Auftreten daraufhin analysiert inwiefern die Ankunftshäufigkeit und im Vergleich zur vorherigen Perioden ansteigt.

\cite{ebina2011real}
A real-time burst detection method

Efficient Elastic Burst Detection in Data Streams 
\cite{Zhu:2003:EEB:956750.956789}

Detecting ‘bursts’ in time series data with Kleinberg’s burst detection algorithm
\cite{kleinberg1}

\subsection{Social Media Analyse}

Wikipedia

Extracting Event-Related Information from
Article Updates in Wikipedia
\cite{10.1007978-3-642-36973-5_22}


\cite{gottschalk2017ongoing},
  Ongoing events in Wikipedia: a cross-lingual case study

\cite{fetahu2015much},
  How much is Wikipedia Lagging Behind News?

\cite{liu2016event}
  Event analysis in social multimedia: a survey

A cloud-enabled automatic disaster analysis system of multi-sourced data streams: An example synthesizing social media, remote sensing and Wikipedia data
\cite{huang2017cloud}

Pinterest
I need to try this?: a statistical overview of pinterest
\cite{gilbert2013need}

Twitter
An evaluation of the run-time and task-based performance of event detection techniques for Twitter
\cite{weiler2016evaluation}
